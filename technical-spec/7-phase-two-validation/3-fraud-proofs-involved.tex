\documentclass[../midgard.tex]{subfiles}
\graphicspath{{\subfix{../images/}}}
\begin{document}

\section{Fraud Proofs in UPLC Evaluation}
\label{s:phase-two-fraud-proofs}

This section details the fraud proofs involved in UPLC evaluation, focusing on single-step verification of state transitions.

\subsection{Types of Fraud Proofs}

The system supports these categories of fraud proofs:

\begin{description}
    \item[Decoding Fraud] Proofs of invalid script byte decoding:
        \begin{itemize}
            \item Invalid byte format
            \item Size limit violations
            \item Reference resolution failures
        \end{itemize}
    
    \item[Execution Fraud] Proofs of invalid UPLC execution:
        \begin{itemize}
            \item Resource limit violations
            \item Invalid state transitions
            \item Incorrect execution results
        \end{itemize}
    
\end{description}

\subsection{Single-Step Verification}

The fraud proof system verifies individual state transitions:

\begin{equation*}
\begin{split}
    \forall s_1, s_2 \in \text{CEKState}: & \text{ claimed\_transition}(s_1 \rightarrow s_2) \text{ valid } \iff \\
    & \text{compute\_next\_state}(s_1) = s_2
\end{split}
\end{equation*}

To prove a violation:
\begin{itemize}
    \item The operator provides the claimed before state ($s_1$) and after state ($s_2$)
    \item The challenger computes the actual next state from $s_1$
    \item If the computed state differs from $s_2$, the transition is invalid
    \item No additional trace information is required
\end{itemize}

\subsection{Proof Data Structure}

The fraud proof structure is minimal:

\begin{equation*}
    \text{FraudProof} \coloneq \left\{
    \begin{array}{ll}
        \text{step\_number} : & \mathbb{N} \\
        \text{before\_state} : & \text{CEKState} \\
        \text{claimed\_after\_state} : & \text{CEKState} \\
        \text{actual\_after\_state} : & \text{CEKState}
    \end{array} \right\}
\end{equation*}

\subsection{Verification Process}

The verification process is straightforward:

\begin{enumerate}
    \item Verify the before state matches the operator's claim
    \item Compute one step from the before state
    \item Compare computed result with operator's claimed after state
    \item If they differ, the fraud proof is valid
\end{enumerate}

\subsection{Security Considerations}

The single-step verification approach provides several benefits:

\begin{itemize}
    \item Minimal proof size
    \item Constant-time verification
    \item No complex challenge periods needed
    \item Deterministic outcomes
\end{itemize}

Security guarantees include:

\begin{itemize}
    \item No false positives (valid transitions cannot be proven invalid)
    \item No trace reconstruction required
    \item Immediate verification of claims
    \item Protection against computational waste attacks
\end{itemize}

\end{document}
