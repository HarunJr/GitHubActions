\documentclass[../midgard.tex]{subfiles}
\graphicspath{{\subfix{../images/}}}

\begin{document}

\section{Single-threaded state machine}
\label{h:single-threaded-state-machine}

\subsection{Utxo representation}
\label{h:single-threaded-state-machine-utxo-representation}

Each single-threaded state machine's current state is represented in the blockchain ledger as a single utxo:
\begin{itemize}
    \item The spending validator of the utxo defines the possible transitions out of the current state.
    \item The utxo value contains a thread token corresponding to the state machine.
    \item The datum contains the output that the machine emitted upon entering the current state.
\end{itemize}

Each state transition of the state machine is executed via a separate blockchain transaction:
\begin{itemize}
    \item The state machine's thread token must be unique within the transaction context.\footnote{This avoids double-satisfaction issues during onchain validation of state transitions, as any transition's before and after states can be uniquely identified in any transaction.
      However, the state machine's thread token does \emph{not} need to be globally unique across the blockchain ledger --- it may be desirable to run several machines simultaneously, evolving their states in independent transaction chains.}
    \item The before-state is represented by the transaction input that contains the thread token.
    \item The after-state is represented by the transaction output that contains the thread token.
    \item The state transition's input is represented by the redeemer provided to the before-state's spending validator.
      If the spending validator defines several state transitions, the redeemer selects one for the transaction.
\end{itemize}

\subsection{Minting policy}
\label{h:single-threaded-state-machine-minting-policy}

The thread token's minting policy defines the state machine's initial states, initialization procedure, final states, and termination procedure.
It also implicitly defines the state machine's subgraph of reachable states, as each initial state's spending validator defines the transitions out of that state and, inductively, all further transitions out of the resulting states.
Redeemers:

\begin{description}
    \item[Initialize.] Mint the thread token and send it to the spending validator address of one of the state machine's initial states.
      This redeemer receives an initial input that selects the initial state and provides additional information that can be referenced in subsequent state transitions.
    
    The thread token's name should indicate the selected initial state and include a hash of the reference information provided in the initial input.
    If the state machine is deterministic, its thread token name identifies its unique path from initialization to the current state.
    
    \item[Finalize.] Terminate the state machine normally if it is in one of the final states.
      Burn the thread token and (if needed) perform cleanup actions that are universally needed when terminating normally from any final state.
    
    \item[Cancel.] Terminate the state machine exceptionally from any state.
      Burn the thread token and (if needed) perform cleanup actions that are universally needed when terminating exceptionally from any state.
\end{description}

\subsection{Spending validators}
\label{h:single-threaded-state-machine-spending-validators}

If the state machine is non-deterministic, some of its spending validators define multiple possible transitions out of some states.
The redeemers provided to these spending validators select the state transitions out of those states.
Furthermore, the redeemers may provide additional arguments so that the spending validators have the context to decide whether to allow the selected state transitions.

Each spending validator is custom-written to express the specific logic of the possible transitions out of its state.
For each of these transitions, the spending validator must include a condition that verifies the transformation of the input state into the corresponding output state:
\begin{align*}
    \T{verify\_transition}_{ij} &: (\T{Input}_i, \T{Output}_j, ..\T{Args}_j) -> \T{Bool} \\
    \T{verify\_transition}_{ij}\T{(i, o, ..args)} &\coloneq
        \Bigl( \T{transition}_{ij}\T{(i, ..args) \equiv \T{o}} \Bigr) \\
    \T{transition}_{ij} &: (\T{Input}_i, ..\T{Args}_j) -> \T{Output}_j
\end{align*}

\subsection{Compilation}
\label{h:single-threaded-state-machine-compilation}

Each spending validator can be either statically or dynamically parametrized on the spending validators into which its outbound state transitions lead.

\begin{itemize}
    \item Static parametrization is preferred for state transitions that are expected to occur more frequently in typical executions of the state machine.
      A fancy way of expressing this is that state transitions should be statically parametrized along the maximally weighted acyclic subgraph of the state graph.
    \item Dynamic parametrization should be used for all other state transitions because statically parametrizing them would cause circular compilation dependencies on more preferred state transitions.
\end{itemize}

Dynamic parametrization means that the spending validator requires a reference input that indicates the addresses of the spending validators on which it dynamically depends.
This reference input is crucial for the integrity of the state machine's state graph, so secure governance mechanisms should control its creation/modification.

\subsection{Example}
\label{h:single-threaded-state-machine-example}

Consider a simplified model of the git pull-request (PR) workflow, with the following states:
\begin{figure}[ht] % place the figure ’here’ or at the page top
    \centering % center the figure
    \begin{tikzpicture}[node distance=2cm]    
        \node (s1) [initialstate] {Draft};
        \node (s2) [state, right of = s1, above of = s1]
            {Testing};
        \node (s3) [state, right of = s2, below of = s2]
            {Review};
        \node (s4) [state, right of = s1, below of = s1]
            {Changes requested};
        \node (s5) [finalstate, right of = s3, below of = s3]
            {Approved};

        \node (sA) [alphastate, left of = s1, below of = s1,
                       xshift = -2cm] {};
        \node (sZ) [omegastate, right of = s5, above of = s5,
                       yshift = 1.5cm] {};
        
        \path[every node/.style={font=\sffamily\small}] [arrow]
            (sA) edge [bend right = 20] (s1)
                 node [right, xshift = 0.7cm, yshift = 0.7cm]
                      {Initialize}
            (s1) edge [loop left]
                 node [anchor = west, left,
                       xshift = 0.5cm, yshift = 0.7cm]
                      {Update} (s1)
            (s1) edge [bend left = 20]
                 node [anchor = west, left] {Request review} (s2)
            (s2) edge [bend left = 20]
                 node [anchor = east, right] {Fail} (s1)
            (s2) edge [bend left = 20]
                 node [anchor = east, right] {Pass} (s3)
            (s3) edge [bend left = 20]
                 node [anchor = east, right] {Changes} (s4)
            (s3) edge [bend left = 20]
                 node [anchor = east, right] {Approve} (s5)
            (s4) edge [bend left = 20]
                 node [anchor = west, left] {Respond} (s3)
            (s4) edge [bend left = 20]
                 node [anchor = west, left] {Accept} (s1)
            (s5) edge [bend right = 20]
                 node [anchor = west, left,
                       xshift = 0.2cm, yshift = 0.5cm]
                      {Merge} (sZ);
    \end{tikzpicture}
    \caption{State diagram for a simplified git PR workflow.}
    \label{fig:single-threaded-state-machine-example}
\end{figure}

\begin{description}
    \item[Draft (initial state).] The drafter is implementing a feature or bug fix in a repository branch.
      Transitions:
        \begin{description}
            \item[Update.] The drafter updates the branch by adding some git commits.
              Next state: Draft.
            \item[Request review.] The drafter requests a review for the branch.
              Next state: Testing.
        \end{description}
    \item[Testing.] The test suite is executing.
      Transitions:
        \begin{description}
            \item[Fail.] The branch fails its test suite.
              Next state: Draft.
            \item[Pass.] The branch passes its test suite.
              Next state: Review.
        \end{description}
    \item[Review.] The reviewer is deciding whether the branch should merge into the main branch.
      Transitions:
        \begin{description}
            \item[Approve.] The reviewer approves the branch to be merged into the repository's main branch.
              Next state: Merged.
            \item[Changes.] The reviewer requests some changes to the branch.
              Next state: Changes requested.
        \end{description}
    \item[Changes requested.] The drafter is considering the reviewer's feedback.
      Transitions:
        \begin{description}
            \item[Respond.] The drafter responds to the reviewer, arguing that changes are not required.
              Next state: Review.
            \item[Accept.] The drafter accepts the reviewer's change requests.
              Next state: Draft.
        \end{description}
    \item[Approved (final state).] The branch is merged into the main branch.
      Transitions:
        \begin{description}
            \item[Merge.] The state machine terminates normally from the final state.
              The branch is merged into the main branch and then deleted.
        \end{description}
\end{description}

The onchain state machine representation of the above git PR model uses one minting policy and five spending validators.
The minting policy:

\begin{itemize}
    \item Defines Draft as the initial state.
    \item Assigns the state machine a token name corresponding to the commit hash of the base branch of the PR.\footnote{In this simplified model, the base branch cannot be changed for a PR.}
    \item Defines Merged as the final state.
    \item Updates the state of the main branch when the PR branch is merged.
\end{itemize}

The spending validators define the transitions out of their corresponding states.
The state datum types include the PR's current commit hash and other information relevant to their outbound transitions.
For example, the spending validator for the Draft state has redeemers to validate two state transitions:
\begin{description}
    \item[Update.] Go to the Draft state.
      Update the PR commit hash and resolve any accepted change requests that the update addresses.
    \item[Request review.] Go to the Testing state.
      Ensure that no accepted change requests remain.
\end{description}

We can also add a Cancel redeemer to every spending validator and the minting policy.
In the git PR workflow, this state transition would reflect the fact that a PR can be closed at any time.
Some additional logic may be needed in each of these redeemers to properly dispose of the PR after it is closed.

\end{document}
